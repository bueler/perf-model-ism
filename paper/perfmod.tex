\documentclass[twocolumn,letterpaper]{igs}
%\documentclass[twocolumn]{igs}  % A4 paper needs no option
%\documentclass[review,letterpaper]{igs}

\usepackage{verbatim,xspace,amsmath,amssymb,bm,multirow,dashbox}
\usepackage{tikz}
\usetikzlibrary{arrows}
\usepackage{igsnatbib,lineno}

\usepackage[kw]{pseudo}
\pseudoset{left-margin=0mm,topsep=5mm,idfont=\texttt}

\usepackage{hyperref}
\hypersetup{pdfauthor={Ed Bueler},
            pdfcreator={pdflatex},
            colorlinks=true,
            citecolor=purple,
            linkcolor=red,
            urlcolor=blue,
            }

% math macros
\newcommand\bb{\mathbf{b}}
\newcommand\bc{\mathbf{c}}
\newcommand\bbf{\mathbf{f}}
\newcommand\bg{\mathbf{g}}
\newcommand\bn{\mathbf{n}}
\newcommand\bq{\mathbf{q}}
\newcommand\bu{\mathbf{u}}
\newcommand\bv{\mathbf{v}}
\newcommand\bx{\mathbf{x}}
\newcommand\by{\mathbf{y}}

\newcommand\bA{\mathbf{A}}
\newcommand\bF{\mathbf{F}}
\newcommand\bH{\mathbf{H}}
\newcommand\bN{\mathbf{N}}
\newcommand\bQ{\mathbf{Q}}
\newcommand\bU{\mathbf{U}}
\newcommand\bV{\mathbf{V}}
\newcommand\bW{\mathbf{W}}
\newcommand\bX{\mathbf{X}}

\newcommand\bzero{\bm{0}}

\newcommand{\Div}{\nabla\cdot}
\newcommand\eps{\epsilon}
\newcommand{\grad}{\nabla}
\newcommand{\ip}[2]{\ensuremath{\left<#1,#2\right>}}
\newcommand\lam{\lambda}
\newcommand\lap{\triangle}
\newcommand\RR{\mathbb{R}}
\newcommand{\half}{\tfrac{1}{2}}

\newcommand{\Divx}{\nabla_\bx \cdot}
\newcommand{\gradx}{\nabla_\bx}

\newcommand{\rhoi}{\rho_{\text{i}}}
\newcommand{\pp}{{\text{p}}}
\newcommand{\qq}{{\text{q}}}
\newcommand{\rr}{{\text{r}}}

\newcommand{\mR}{R^{\bm{\oplus}}}
\newcommand{\iR}{R^{\bullet}}

\newcommand{\sold}{s_{\text{o}}}
\newcommand{\told}{t_{\text{o}}}

\newcommand{\byold}{\by_{\text{o}}}

\newcommand{\onecol}[1]{\includegraphics[width=86mm]{#1}}
\newcommand{\onecolless}[1]{\includegraphics[width=80mm]{#1}}
\newcommand{\onecolthin}[1]{\includegraphics[width=75mm]{#1}}

%\newcommand{\twocol}[1]{\includegraphics[width=178mm]{#1}}
\newcommand{\twocol}[1]{\includegraphics[width=165mm]{#1}}
\newcommand{\twocolless}[1]{\includegraphics[width=140mm]{#1}}

\newcommand{\ds}{\displaystyle}

\begin{document}

\title[Performance models for high-resolution ice sheet simulations]{Letter: Performance models for \\ high-resolution ice sheet simulations}

\abstract{High-resolution numerical ice sheet models, with horizontal resolutions of a few kilometers or less, compute evolving ice geometry and velocity fields using various stress-balance approximations and boundary conditions.  These models usually require many short, stability-limited time steps because ice thickness is updated following each velocity solution.  Practical performance in the high-spatial-resolution limit is degraded by the stability conditions of such explicit time-stepping.  In this short note, which considers the shallow ice approximation and Stokes models as end members, we attempt to clarify numerical model performance by quantifying the simulation cost per model year in terms of the number of degrees of freedom.  The performance of coming generations of implicit time-stepping numerical models is then assessed.}

\author{Ed Bueler}

\affiliation{Dept.~Mathematics and Statistics, University of Alaska Fairbanks, USA \\
E-mail: \emph{\texttt{elbueler\@@alaska.edu}}}

%\keywords{}

\maketitle

\sectionsize

\section{Introduction}

Numerical ice sheet (glacier) models with evolving ice geometry are now in common use for scientific questions such as quantification of future sea level rise from changes in the Antarctic \citep{Seroussietal2020} and Greenland \citep{Goelzeretal2020} ice sheets, interpretation of the paleoglacial record \citep{Weberetal2021}, and quantification of long-term glacial erosion \citep{SeguinotDelaney2021}, among other applications.  It is generally accepted that horizontal mesh (grid) cells must be smaller than about 10 km in order to generate valid results.  Whether using local mesh refinement \citep[for example]{Hoffmanetal2018} or not, resolutions of a few kilometers to one kilometer \citep{SeguinotDelaney2021} or less \citep{Aschwandenetal2019} are increasingly used at ice sheet scale.  With currently-available computational resources, sub-kilometer meshes should become common for long-duration, whole ice sheet studies.

Current-generation ice sheet models use a variety of stress balances, from simple approximations like the shallow ice approximation (SIA), through ``hybrid'' \citep{Robinsonetal2022,Winkelmannetal2011} and higher-order balances, to the non-shallow Stokes approximation.  With very few exceptions, however, current models alternate between solving the stress balance for velocity, over the geometry determined by the previous time step, and then updating the geometry, i.e.~the ice thickness and/or surface elevation.  The geometry-update operation, using the mass continuity or surface kinematical equations \citep{GreveBlatter2009}, is where interaction between the ice sheet and the climate is (primarily) simulated, especially through surface mass balance, sub-shelf (basal) mass balance, and calving processes.  Such schemes are using \emph{explicit} time-stepping for the coupled mass and momentum system which describes ice dynamics.

For simpler partial differential equation problems, the conditional stability of explicit time-stepping schemes is well-understood \citep{LeVeque2007}.  The stability limitations of explicit SIA models have also been understood for some time \citep{HindmarshPayne1996}, but recent studies have focussed on the time-step limits of hybrid, higher-order, and Stokes dynamics models \citep{Chengetal2017,Robinsonetal2022}, or on lengthening the steps \citep{LofgrenAhlkronaHelanow2021}.

However, \emph{implicit} time-stepping should also be considered.  Here the velocity and geometry (thickness or surface elevation) are updated simultaneously by solving a coupled mass-momentum problem.  While implicit time-stepping is a routine solution process for certain simpler partial differential equations \citep[for example]{LeVeque2007}, for ice sheets the implicit step must simultaneously compute the velocity and the \emph{domain on which the velocity is defined}.  The problem is of free-boundary type in map-plane (horizontal) directions \citep{SchoofHewitt2013}, as well as in the easily-resolved vertical direction.  An implicit strategy has been demonstrated at high-resolution in the simplest frozen-base, isothermal SIA case \citep{Bueler2016}.  The same reference shows how the steady-state problem \citep{JouvetBueler2012} can often be solved; steady-state solutions are an implicit time step of infinite duration.  Early work applying a semi-implicit time step using Stokes dynamics \citep{WirbelJarosch2020} illuminates some of the techniques, and difficulties, needed to make such a free-boundary problem work with a membrane-stress-resolving balance.

This is the context in which we will relate time-stepping and stress-balance choices to overall computational effort.  The reader is assumed to have access to the SIA and Stokes stress-balance equations \citep{GreveBlatter2009,SchoofHewitt2013}, regarded here as end members of current-usage stress-balance approximations \citep{Robinsonetal2022}, and familiarity with the mass continuity and surface kinematical equations \citep{GreveBlatter2009}.

However, the manner of presenting the mass continuity equation should first be examined.  For an incompressible ice sheet with thickness $H(t,\bx)$, vertically-averaged velocity $\bu(t,\bx)$, and climatic-basal mass balance $a(t,\bx)$, this equation says
\begin{equation}
\frac{\partial H}{\partial t} + \Divx \left(\bu H\right) = a, \label{eq:masscontinuity}
\end{equation}
where $\bx=(x,y)$ denotes horizontal coordinates.

Equation \eqref{eq:masscontinuity} suggests that ice sheets change geometry in an essentially advective manner, but this appearance is deceiving, or at least over-simplified, regarding the growth of numerical instabilities.  This is because ice flows dominantly downhill.  Thus, as addressed by linearized analysis \citep{Chengetal2017,Robinsonetal2022}, when thickness perturbations grow unstably under explicit time-stepping, i.e.~with too large a step, they do so by a mix of advective and diffusive mechanisms.

Let $s(t,\bx)=H(t,\bx)+b(\bx)$ denote the surface elevation for bed elevation $b(\bx)$.  If the numerical thickness is perturbed then the velocity $\bu$ often responds by growing in the downhill direction $-\gradx s$, a direction often correlated to $-\gradx H$.  In membrane-stress-resolving models this happens through the nonlocal solution of the stress balance; the driving stress is along $\gradx s$.  Equation \eqref{eq:masscontinuity} has velocity $\bu$ which is really a non-local function ``$\bu(H,\grad_x s)$''; a solution of the stress balance equations is required to evaluate this function.  Equivalently, ice sheet flow has no characteristic curves because the velocity $\bu$ actually depends indirectly on the gradient of thickness.

Thus the mass continuity equation is substantially, though not wholely, diffusive.  The diffusive nature dominates in the small-parameter limit which generates the SIA, in which a revised mass continuity equation holds:
\begin{equation}
\frac{\partial H}{\partial t} = \Divx \left(d\, \gradx s \right) + a, \,  \label{eq:siamasscontinuity}
\end{equation}
where $d = C H^{n+2} |\gradx s|^{n-1}$ is the nonlinear diffusivity, $C = 2 A (\rho g)^n/(n+2)$ is a positive constant, $A$ is the (isothermal) ice softness, $\rho$ is the ice density, $g$ is gravity, and $n$ is the Glen exponent in the flow law \citep{GreveBlatter2009}.

Unlike equation \eqref{eq:masscontinuity}, equation \eqref{eq:siamasscontinuity} does not hold for all models.  Regardless, the same diffusivity $d$, an essentially geometric quantity, can be computed.  For grounded ice sheets large values of $d$ will indicate locations of unstable mode growth if explicit time-steps are chosen too large.  

The present-day understanding of ice sheet model time stepping is complicated.  While the author expects that time-stepping ice sheet models will eventually become faster when implicit time-stepping and advanced solver techniques like multigrid \citep{Briggsetal2000} are applied, this belief should be inspected quantitatively to the extent possible.  A simplified performance model can expose the most important considerations and trade-offs.

\section{Performance models}

We analyze simulation performance using $n$, the number of mesh/grid nodes in the horizontal map-plane.  (Table \ref{tab:notation} lists our notation.)  A numerical ice sheet model uses $n$ ice thickness or surface elevation variables, the number of \emph{degrees of freedom}.  It requires $O(n)$ memory to store the instantaneous model state, including thermodynamical variables for example, if we assume that the vertical mesh/grid has an \emph{a priori} bounded resolution.\footnote{This bounded-vertical-nodes assumption reflects common usage \citep[for example]{Aschwandenetal2019,Brinkerhoffetal2017,Hoffmanetal2018}, but it may hold only in an approximate sense \citep{IsaacStadlerGhattas2015}.}  The amount of fast memory needed by a running ice sheet simulation is also $O(n)$ if prior states are discarded or transferred to storage.

\begin{table*}[ht]
\begin{tabular}{cll}
\emph{name} & \emph{meaning} & \emph{units} \\ \hline
$\alpha$    & one fixed-geometry Stokes velocity solution requires $O(n^{1+\alpha})${\large \strut} work\\
$\beta$     & one implicit SIA geometry-update requires $O(n^{1+\beta})$ work \\
$\gamma$    & one implicit Stokes geometry-update plus velocity solution requires $O(n^{1+\gamma})$ work \\
$D$         & representative geometric (shallow ice approximation) diffusivity of an ice sheet & $\text{km}^2 \text{a}^{-1}$ \\
$L$         & width of simulation domain & km \\
$n$         & degrees of freedom: number of nodes in the horizontal (flowline or map-plane) mesh \\
$q$         & time steps per model year needed to resolve modeled climate interactions & $\text{a}^{-1}$ \\
$T$         & simulation duration in model years & a \\
$\Delta t$  & length of time step in model years & a \\
$U$         & representative horizontal ice velocity & $\text{km}\,\text{a}^{-1}$ \\
$\Delta x$  & representative width (diameter) of map-plane mesh cells & km
\end{tabular}
\caption{Notation for performance modeling.  Parameters $\alpha,\beta,\gamma,n$ are pure numbers.}
\label{tab:notation}
\end{table*}

Ice sheet models also have $O(n)$ velocity variables.  However, these are not state variables because a very-viscous stress balance can recompute velocity as a function of (true) state variables.

The model domain is of (horizontal) width $L>0$.  If $\Delta x$ is a representative diameter of the map-plane mesh cells then $\Delta x = O(L n^{-1/2})$; there are $O(\sqrt{n})$ mesh cells in each horizontal dimension.  In flow-line models, not our primary motivation but included for completeness, $\Delta x = O(L n^{-1})$.
 
Ice sheet models involve climate interactions, especially via surface mass balance, which are dominated by an annual cycle and longer time-scale changes.  Let $q$ be the number of ice-dynamical time steps per year needed to capture this climate interaction; typical values $q=0.1 \,\text{a}^{-1}, 1 \,\text{a}^{-1}, 12 \,\text{a}^{-1}$ correspond to decadal, yearly, and monthly flow/climate interaction frequency, respectively.  Energy balance and degree-day schemes for computing surface mass balance generally have much shorter time scales, but here $q$ only describes the frequency on which ice geometry is updated from an ice velocity solution.

Current-technology ice sheet models mostly use explicit time-stepping which is only conditionally stable \citep{LeVeque2007}.  For the spatial resolutions used in present-day scientific applications, we observe that maintenance of explicit time-stepping stability requires time steps substantially shorter than $1/q$ model years.  Unconditionally-stable implicit schemes also have a maximum time step restriction, namely $1/q$ itself, reflecting the simulation purpose and not the maintenance of stability.

For an explicit SIA model the well-known stability restriction is $\Delta t < O(D^{-1} \Delta x^2)$ \citep{Bueleretal2005,HindmarshPayne1996} where $D$ is a representative diffusivity value from the diffusivity formula in equation \eqref{eq:siamasscontinuity}.

For Stokes dynamics the stability of explicit time-stepping is largely unexplored in any precise sense \citep[compare][]{Chengetal2017}.  We propose that an advective restriction $\Delta t < O(U^{-1} \Delta x)$, for some representative horizontal velocity scale $U$, computed from simulated ice velocities, represents the \emph{optimistic} paradigm.  The corresponding \emph{pessimistic} paradigm enforces $\Delta t < O(D^{-1} \Delta x^2)$, using diagnostic diffusivity values computed in the SIA manner.

Explicit time-stepping with hybrid and higher-order schemes is better-studied than for Stokes dynamics, especially over horizontal resolutions relevant to whole ice sheets.  Some hybrid schemes apply the pessimistic time step as an adaptive restriction \citep{Winkelmannetal2011}.  The optimistic time step is supported for a certain higher-order scheme \citep[``DIVA'';][equation (52)]{Robinsonetal2022}, but practical Greenland simulations in the same work actually suggest $\Delta t = O(\Delta x^{1.6})$ (Figure 3(a)).

Regardless of time-step stability, solution of the momentum balance equations requires effort.  In general this requires a nontrivial solution of spatial partial differential equations.  For the Stokes model, in particular, we suppose one such solution requires $O(n^{1+\alpha})$ work, namely floating point operations in our simple performance model, where $\alpha\ge 0$ depends on the design of the stress balance solver.  Naive solution using direct linear algebra might involve $\alpha=3$, or $\alpha \ge 2$ when exploiting sparsity.  However, application of a multigrid method, such as the \cite{IsaacStadlerGhattas2015} solver specifically adapted to the geometry and flow law applicable to ice sheets, suggest that $\alpha$ can potentially be close to zero, with the important caveat that the constant implied in ``$O(n^{1+\alpha})$'' is very large.  Multigrid methods have also been applied to higher-order stress balance equations \citep{BrownSmithAhmadia2013}; again $\alpha \approx 0$ is a credible aspiration.  On the other hand, the SIA velocity computation is a trivialization of the Stokes problem; here each velocity is computed in $O(1)$ work, and the velocity solution in $O(n)$ work.

The stress-balance scheme determines the velocity appearing in equation \eqref{eq:masscontinuity}.  From the velocity field, regardless of the stress balance source, an explicit time-stepping scheme computes the updated ice thickness using $O(n)$ work.

On the other hand, an unconditionally-stable implicit geometric-update scheme in the SIA, as demonstrated by \cite{Bueler2016}, FIXME $O(n^{1+\beta})$ work; for Stokes this is a different task of solving coupled equations and we merely write $O(n^{1+\gamma})$ for now

FIXME $\Delta t^{-1}$ time steps per model year (if $\Delta t$ is in model years)

For such an implicit scheme, our stability model is that there is no restriction in the fine-resolution limit ($\Delta x \to 0$).  For example, the steady-state SIA solver in \cite{Bueler2016} uses $\Delta t=+\infty$, with $\Delta t = 100 \,\text{a}$ as a recovery strategy in the case of solver non-convergence.

At each time step of an ice sheet simulation the geometry is updated, for example the old surface elevation values are replaced by new ones.  For an explicit stepping scheme the work in this update is small, specifically $O(n)$, because, once the ice velocity is computed from fixed geometry by the chosen momentum balance model, each grid point requires $O(1)$ work to update.

Table \ref{tab:performancemodel} summarizes our performance models.

\newcommand{\oo}[1]{{\LARGE \strut} \displaystyle O\left(#1\right)}
\setlength{\tabcolsep}{5pt}
\renewcommand{\arraystretch}{1.5}
\begin{table*}[ht]
\begin{tabular}{llll}
\emph{time-stepping} & \emph{dynamics} & \emph{flops per model year \, $\left[\text{1D}\right]$} & \emph{pessimistic \, $\left[\text{1D}\right]$} \\ \hline
explicit & SIA    & $\oo{\frac{D}{L^2} n^2}$ \, $\left[n^3\right]$ \\
explicit & Stokes & $\oo{\frac{U}{L} n^{1.5+\alpha}}$ \, $\left[n^{2+\alpha}\right]$ & $\oo{\frac{D}{L^2} n^{2+\alpha}}$ \, $\left[n^{3+\alpha}\right]$ \\
implicit & SIA    & $\oo{q\, n^{1+\beta}}$ \\
implicit & Stokes & $\oo{q\, n^{1+\gamma}}$
\end{tabular}
\caption{Asymptotic scaling of computational work, floating point operations per model year, for time-stepping numerical ice sheet simulations, as the degrees of freedom $n$ goes to infinity.  Note 1D refers to flow-line geometry; corrected power on $n$ for 1D is in square brackets.  See Table \ref{tab:notation} for remaining notation.}
\label{tab:performancemodel}
\end{table*}

We assume that the simulation is over a time interval $[0,T]$ where $T>0$ is the \emph{duration} in model years.  For explicit methods note that $\Delta t = T/m$ for $m$ actual, stability-limited time steps in the simulation.  That is, we may define $m=T/\Delta t$ as the number of actual time steps taken.

Combining our observations about the scaling of $\Delta x$ with $n$, for explicit schemes we have
\begin{equation}
\Delta t^{-1} = \frac{m}{T} > O\left(\frac{D n^2}{L^2}\right)
\end{equation}
in the 1D SIA and pessimistic-Stokes cases and
\begin{equation}
\Delta t^{-1} > O\left(\frac{D n}{L^2}\right)
\end{equation}
in the corresponding 2D cases.  (Recall that the power on $n$ is lower in 2D cases because $\Delta x = L/\sqrt{n}$.)  In the optimistic-Stokes case these estimates become
\begin{equation}
\Delta t^{-1} > O\left(\frac{U n}{L}\right), O\left(\frac{U \sqrt{n}}{L}\right)
\end{equation}
in 1D and 2D respectively.

This estimate of $\Delta t^{-1}$ is multiplied by the per-step expense to give a work estimate for each model year in a simulation.  The results are shown in the ``explicit'' rows of Table \ref{tab:performancemodel}.  Note that 2D simulationsare much more expensive for a given resolution $\Delta x$ because $n$ is so much larger; $n = O(\Delta x^{-1})$ in 1D while $n = O(\Delta x^{-2})$ in 2D.

For unconditionally-stable implicit methods, however, $\Delta t$ does not need scale with grid resolution; it has fixed value $\Delta t = 1/q$ model years.  That is, the time steps are determined only by the need to resolve climatic interactions.

We suppose that the work per geometry-update solution of an implicit SIA simulation is $O(n^{1+\beta})$ for some $\beta>0$.  The \cite{Bueler2016} geometry-update method is fundamentally not a multigrid method, and it has FIXME WHAT $\beta$?

For an implicit Stokes simulation the geometry/velocity solve is assumed to be, in the absense of constraining research, $O(n^{1+\gamma})$ for some $\gamma>0$ to be determined.  The nMCD method proposed in [Bueler 2022] is expensive but multigrid, so it may turn out that $\gamma$ is small.

Noting that $n$ is the number of horizontal grid points, the work estimate for an implicit time step may not depend on the dimension, but in fact it is likely that $\alpha,\beta,\gamma$ each depend on dimension in some manner; this should be informed by further research.  Furthermore, the estimate reported in Table \ref{tab:performancemodel} must be multiplied by a potentially very large scheme-dependent constant.

\section{Discussion and Conclusion}

Existing multigrid methods for the velocity solution process show $\alpha$ is close to zero.  That is, the evidence shows that $\alpha$ is small for well-engineered Stokes \citep{IsaacStadlerGhattas2015} and higher-order \citep{BrownSmithAhmadia2013} solvers.

The promise of multigrid methods for the implicit, coupled geometry-update plus velocity solution problems is to also make $\beta,\gamma$ smaller.  However, this aspiration is long-term because work is barely started on it.

%         References
\bibliography{perfmod}
\bibliographystyle{igs}

\end{document}
